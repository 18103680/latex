\documentclass[12pt,calibri]{article}
\usepackage{graphicx}
\usepackage{blindtext}
\title{\bfseries \small PERAMALAN PERSEDIAAN MATERIAL DI PETROCHINA INTERNATIONAL 
JABUNG Ltd DENGAN METODE MOVING AVERAGE}

\author{Laporan Magang}
\date{}

\begin{document}

\maketitle

\begin{figure}[h]
    \centering
    \includegraphics[width=0.4\textwidth]{unja.png}
\end{figure}

\begin{center}
    {\bfseries  Kevin Synagogue Panjaitan} \\
    {\bfseries   F1C221058}
\end{center}
\vspace{10pt}

\begin{center}
    {\bfseries \large Program Studi Matematika }\\
    {\bfseries \large Jurusan Ilmu Pengetahuan Alam}\\
    {\bfseries \large Falkultas Sains Dan Teknologi }\\
    {\bfseries \large Universitas Jambi}
\end{center}
\vspace{10pt}

% \begin{center}
%     {\bfseries \large Falkultas Sains Dan Teknologi }\\
%     {\bfseries \large Universitas Jambi}
% \end{center}

\newpage
\begin{center}
    \bfseries\Large Ringkasan 
\end{center}
\hspace*{1cm} \blindtext

\newpage
\begin{center}
    \bfseries \small PERAMALAN PERSEDIAAN MATERIAL DI PETROCHINA IN-
    TERNATIONAL JABUNG Ltd DENGAN METODE MOVING AVERAGE

\end{center}

\begin{center}
    \large Laporan Magang

\end{center}
\begin{center}
    \small Diajukan seagai salah satu syarat untuk menyelesaikan mata kuliah Magang
    pada Program Studi Matematika

\end{center}
\vspace*{10pt} 

\begin{figure}[h]
    \centering
    \includegraphics[width=0.4\textwidth]{unja.png}
\end{figure}

\begin{center}
    {\bfseries  Kevin Synagogue Panjaitan} \\
    {\bfseries   F1C221058}
\end{center}
\vspace{10pt}
\begin{center}
    {\bfseries  Dosen Pembimbing ;} \\
    {\bfseries   Niken Rarasati }
\end{center}
\vspace{10pt}
\begin{center}
    {\bfseries \large Program Studi Matematika }\\
    {\bfseries \large Jurusan Ilmu Pengetahuan Alam}\\
    {\bfseries \large Falkultas Sains Dan Teknologi }\\
    {\bfseries \large Universitas Jambi}
\end{center}
\newpage

\begin{center}
    \bfseries\Large Halaman Pernyataan
\end{center}
\hspace*{1cm} Penulis menyatakan bahwa laporan magang ini merupakan hasil karya
sendiri. Sepanjang pengetahuan penulis tidak terdapat karya atau pendapat
yang ditulis atau diterbitkan orang lain kecuali sebagai acuan atau kutipan
dengan mengikuti tata penulisan karya ilmiah yang telah lazim.

Tanda tangan yang tertera dalam halaman pengesahan ini adalah asli.
Jika tidak asli, penulis siap menerima sanksi sesuai dengan peraturan yang
berlaku.

\vspace*{40pt}
\begin{flushright}
    \small Jambi, Desember 2024
\end{flushright}
\begin{flushright}
    \small Yang menyatakan,
\end{flushright}
\vspace*{40PT}
\begin{flushright}
    \small {Kevin Synagogue Panjaitan}\\
  \normalfont \small  F1C221058
\end{flushright}

\newpage

\begin{center}
    \bfseries\Large Lembar Pengesahan
\end{center}

\begin{center}
    \bfseries \small PERAMALAN PERSEDIAAN MATERIAL DI PETROCHINA IN-
    TERNATIONAL JABUNG Ltd DENGAN METODE MOVING AVERAGE

\end{center}
\end{document}
